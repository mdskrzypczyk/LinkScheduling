\documentclass{article}
\usepackage[utf8]{inputenc}
\usepackage{amsmath}
\usepackage{graphicx}
\usepackage[a4paper, margin=1in]{geometry}

\title{Literature Study Notes}
\author{mdskrzypczyk }
\date{August 2019}

\begin{document}

\maketitle

\section{Other Notes}
\begin{itemize}
    \item Our problem is not necessarily circuit-switching.  This is because a path on circuit switching is reserved even when no data is flowing.  In what I have done so far, I only reserve a link for a brief period of time, not the whole time the full-length link is being built.  Reserving the whole path for the full entanglement link generation may be one approach to doing so but does not describe the internal coordination by the nodes for generating the entangled link.  The occupied timespan can be optimized though based on the available slots.
    \item My approach resembles more closely the packet-switched network where only the links have schedules and are only occupied for the duration a link needs to be built across it.
\end{itemize}

\section{Time Slot Allocation for Real-Time Messageswith Negotiable Distance Constraints}
Libin Dong, Rami Melhem, Daniel Moss

\subsection{Thoughts}
\begin{itemize}
    \item Could be useful for repeating schedules to get applications $n$ pairs.
\end{itemize}

\subsection{Notes}
\begin{itemize}
    \item (Real-time traffic) Periodic Scheduling Problem - Events and activities to be identically repeated at a constant rate.  Periodic phenomena may arise either naturally, or as the consequence of imposed constraints for the reasons of convenience or efficiency.  In particular this may occur whenever a finite set of actions must be repeated with an infinite time horizon.
    \begin{itemize}
        \item Rate-monotonic-scheduling (RMS)
        \item Earliest-deadline-first (EDF)
    \end{itemize}
    \item Periodic tasks/messages must satisfy a timing constraint requirement relative to the finishing time of a previous instance (ref 5), defined as the distance constraint system model
    \begin{itemize}
        \item Example - Along a network link of bandwidth $B$ a video stream may require data transmission at a rate of $R \times B$ where $R$ is a proportion of full bandwidth.  The average distance between the transmission of consecutive frames must be $1/R$ time slots.  In order to maintain human perception condition of video, time interval between two consecutive video frames must not exceed some maximum value, $D$, which is taken as the distance constraint requirement of the message stream.  Scheduling algorithms for periodic model are not applicable to distance constraint model.
        \item Distance constraint scheduling closely related to pinwheel problem ($A=\{a_1,...,a_n$ with $a_1 \leq ... \leq a_n$, find an infinite sequence of symbols such that symbol $i$ occurs within any interval of $a_i$ slots).  Density $\rho(A)=\sum_{i=1}^n \frac{1}{a_i}$ used for schedules, various bounds $\rho_{max}$ based on method.
    \end{itemize}
    \item Average and maximum distance between slots considered the same but this characterization fails to represent real-time applications (example provided).  Can relax a distance constraint for a stream allows finding a schedule.
    \item Paper proposes pre-allocation based scheme for scheduling $n$ message streams with rate and maximum distance constraint requirements.  Rate requirement is critical QoS requirement which cannot be violated while distance constraint is negotiable.  If a distance constraint cannot be satisfied a negotiation for relaxing it may be done and the algorithm terminates if negotiation fails.
    \item Set of streams $\mathcal{M}=\{(A_i, D_i)\}$ where $A_i$ and $D_i$ represent numbers of slots and $A_i \leq D_i$.  Total density factor $\rho(\mathcal{M}) = \sum_{i=0}^n \frac{1}{A_i} \leq 1$ in order to obtain a feasible schedule.
    \item First calculate the size of the schedule $N$ to satisfy above. By taking LCM of all $A_i$ we are guaranteed there is at least one such $N$, it may be large and it is possible to check if there are smaller $N$ that satisfy.  Example calculating this value is provided in Example 2.
    \item Time slot allocation algorithm similar to EDF used.
    \begin{itemize}
        \item Highest priority is assigned to the message stream with earliest deadline.
        \item Main difference is that EDF has fixed ready times and deadlines while this algorithm allows dynamically calculating ready time and deadline based on allocation of the previous instance and distance constraint requirement.
        \item Deadline of next instance calculated so that distance between current instance and next instance does not exceed distance constraint (\emph{forward} direction).
        \item Allocation pattern repeats every $N$ slots so distance between first instance in current template and last instance in previous template should also be constrained (\emph{backward} direction).
        \item Algorithm sets initial distance parameter to $A_i$ and only increases towards $D_i$ when negotiation fails.
        \item Tie-breaking EDF using $\frac{distance}{D_i}$, ie. the message stream that has least flexibility in increasing distance constraint.
    \end{itemize}
    \item Performance evaluation and use for WDMA passive star couplers.
\end{itemize}

\section{Assignment of Segmented Slots EnablingReliable Real-Time Transmission inIndustrial Wireless Sensor Networks}
Dong Yang,Member, IEEE, Youzhi Xu, Hongchao Wang,Member, IEEE,Tao Zheng,Student Member, IEEE, Hao Zhang, Hongke Zhang, and Mikael Gidlund,Member, IEEE

\subsection{Thoughts}
\begin{itemize}
    \item Retransmission strategies might be useful when considering links that may not be able to succeed within their timeslots?
    \item Notion of superframe based on the application. What timeframe could be used for the frame in quantum networks for establishing links?
    \item Routing-ordered slot scheduling may be useful for deciding which order the incoming batch of routes should be satisfied.
    \item Techniques for separating slots to allow retransmision before next hop transmission cannot really be used in quantum networks due to decoherence.
    \item SS/CCA method in FSC involves communicating with other nodes.  May work on smaller WSNs but will not work at long distances when latencies become larger.
    \item Breaking down superframes into dedicated/shared portions may allow second chance at completing a long distance link.  Dedicated/periodic portion could be used for when multiuple pairs are requested and single pairs could go into a dedicated portion of the superframe.
    \item Markov chain model involving the probability of successful transmission may be useful given link establishment is probabalistic.
    \item Packet aggregation is used for ordering the slot assignments, is there a way link aggregation could be employed at this stage?
    \item Scheduling in IWSN and classical network transmission is a waterfall process, you need to hop in one direction.  In quantum networks the long distance link can in principle be built in any order, reverse, random, etc.
\end{itemize}

\subsection{Notes}
\begin{itemize}
    \item TMDA used in wireless sensor networks because it allows internal channel access contentions to be avoided totally and a significant reduction in transmission errors.
    \item None of the existing IWSNs slot-scheduling studies [31]–[33] considered this problem. Although [33] andIEEE 802.15.4e have mentioned the shared slot (SS) scheduling for retransmission in IWSNs, they simply assign some SSs atthe end of the superframe used for retransmission.
    \item The main motivations of this paper are from the challenges, including slot resource constraints, inefficient SS competition, routing-ordered slot scheduling, and rescheduling caused by nk or node failure, encountered in a real IWSN deployment, where the application was the real-time monitoring of welder machines.
    \item Retransmission unpredictable, propose new shared slot competition algorithm called fast slot competition to improve success rate of retransmissions with limited slot resources.
    \item New slot scheduling algorithm Segmented Slot Assignment (SSA) with main purpose to improve retransmission efficiency.
    \item Concept of free node to decrease complexity and cost of rescheduling caused by link or node failure.
    \item Duration of superframe chosen based on the sampling period of IWSN application network.  Used 500 ms and due to slot length in WirelessHART standard of 10ms this means each superframe has 50 slots.
    \item Devices assumed to have simplex radios meaning one slot used for receiving and one slot for transmitting.  Some slots are dedicated to retransmitting erroneous packets as well as periodic messages like keep-alive, need to consider how to use slots in a good way.
    \item Routing-ordered slot scheduling - Consider the order of routes and corresponding slot assignments.
    \item Rescheduling costs bandwidth and time to update all nodes (ref 37 highlight).  Paper defines metrics rescheduling convergence time (time period from a link or node failure to recovery by rescheduling, \#slots) and rescheduling overhead (Rescheduling information issued to the related nodes and is quantified by the number of related nodes).
    \item Superframes broken into a dedicated-slot part and a shared-slot part where dedicated-slot portion used for periodic sample packets and shared-slot part used for retransmission of error packets.
    \item Fast Slot Competition (FSC)
    \begin{itemize}
        \item Current exponential backoff algorithms for CSMA/CA result in shared slot running out of current superframe, aggravated by restraint SS resource.  Aims to improve retransmission success rate in current superframe with limited number of SS.
        \item CCA used in IEEE wireless standards to determine whether a medium is idle. FSC embeds multiple CCA slots into SS for channel sense operation.  In this scheme, when a CCA slot (chosen at random from within the SS) is detected to see that the SS is idle, a preamble reserving the channel is sent immediately.  Competitors see this preamble and wait until the next SS to compete without exponential backoff.
        \item Analysis includes Markov chain formulation and analysis with number of dedicated-slots/shared-slots/probability of successful transmission.
    \end{itemize}
    \item Shared slots are not placed after every dedicated slot due to restricted slot resources.
    \item Slot scheduling based on hops - Schedule transmissions in order based on number of remaining hops.  Also proposes placing Shared Slots between every set of dedicated slots that belong to a specified hop count.  Rescheduling due to link or node failure is complicated (example in figure 9 provided).
    \item Rescheduling with free node concept proposed.  Free nodes are one-hop nodes that have no descendents (figure 1). Rationale of free node concept is that transmission from one-hop nodes to the sink can be done in any segment of the superframe without affecting the slot segmenting scheduling.
    \item Slot-Scheduling Algorithm (SSA)
    \begin{itemize}
        \item
    \end{itemize}
\end{itemize}

\section{An improved algorithm for slot selection in the AEtherealNetwork-on-Chip}
Radu Stefan and Kees Goossens

\subsection{Thoughts}
\begin{itemize}
    \item Paper focuses on slot selection algorithm and ignores path selection, similar to what is done in project
    \item The paper (seems to) focus on scheduling for access to a common communication bus resource between many chips in a network. Not very applicable to our problem.
\end{itemize}

\subsection{Notes}
\begin{itemize}
    \item "The slot selection algorithm is given the set of availableslots on a path has the task of identifying a minimal subsetof slots that provide the required bandwidth and latency."
    \item Bandwidth expressed in words/slot table revolution (?).  Calculated from bandwidth of link, size of a slot table, and required bandwidth.
    \item Maximum latency expressed in units of length in time of a slot obtained by subtracting from latency required by application the latency that is due to network traversal which is computed from the path length.
    \item Previous algorithms address latency and bandwidth requirements separately and is greedy that only takes optimal decisions locally.
    \item Proposed algorithm optimizes according to both criteria and is optimal in that it uses minimal number of time slots.
    \item Build partial solutions.
\end{itemize}

\section{Time Slot Schedule based Minimum Delay Graph in TDMA Supported Wireless Industrial System}
Yonghoon Chung and Ki-hyung Kim, Seung-wha Yoo

\subsection{Thoughts}
\begin{itemize}
    \item Advertising time slot schedules to machines in a network is known as a "provisioning process".
    \item Assumes a fixed time slot schedule for wireless devices, not really applicable to our problem, maybe to network layer?  Could be interesting if a different approach is to install a schedule for building links in the network in a specific order and then a node has to determine which set of hops/links should be used.  But in these types of scenarios how is the underlying time slot schedule derived?
\end{itemize}

\subsection{Notes}
\begin{itemize}
    \item Focuses on graph generation method by considering the schedule of time slot in superframe.
    \item Time Division Multiple Access based MAC protocol used by the data link layer as it is contention free and prevents network collisions.  Makes impossible that more than one node sends packets at the same time.
    \item On top of TDMA, the network layer maintains a graph of paths and secondary paths for guaranteed transmission routes.
    \item Propose a time slot scheduled based graph generation method in the TDMA network for minimizing the end-to-end delay and round-trip delay time.  "Minimum delay graph generation algorithm"
    \item ELHFR algorithm (Shortest Path Graph) is a graph generation method that defines the graph has minimum number of hops from source to destination.  Draws graph using shortest path algorithm.
    \item If graph installed according to the network joining order the graph detoured due to an inefficiency can be generated.  To prevent this, nodes use a BFS tree and distance vector method like AODV (ref 8)
    \item Assume a fixed time slot schedule for transmission in a wireless network and then tries to determine the shortest number of time slots where hops may be used to get message from source to destination.
\end{itemize}


\section{Dynamic Scheduling of Real-Time Messages over an Optical Network}
Cheng-Chi Yu and Sourav Bhattacharya

\subsection{Thoughts}
\begin{itemize}
    \item Idea of (time, wavelength) TWDM scheduling could be used as a base for (time, communication qubit) model?
    \item Hard and soft deadlines can be established based on fidelity requirements?  Hard deadline being the lowest acceptable fidelity.
    \item Ideas for inputs to analysis may be useful, ie. load, proximity of deadline values, relative mix of hard and soft deadline messages, as well as priority levels.
    \item Figure 1 slot schedules resemble our schedules very closely.
    \item Idea for preempting - When a new link request comes in we can filter out links that have a lower priority than the incoming one, schedule the link, and then preempt links that overlap with it when considering the remove priority links.
    \item Can use similar scheduling heuristics based on how many links need to be preempted (want least number) and priority of the link.
\end{itemize}

\subsection{Notes}
\begin{itemize}
    \item Consider dynamic scheduling in a Time-Wave Division Multiplexing (TWDM) transmission schedule where slots are denoted by (time, wavelength)
    \item Time-critical messages have hard and soft deadlines (hard have highest priority and soft are user defined lower priority levels).
    \item Goal is to schedule the messages, all or as many as possible following the priority ordering.
    \item Previous research is in static scheduling policies which cannot adapt to varying traffic conditions or dynamic scheduling for non-realtime traffice.
    \item Performance measured in simulation where input factors include load, proximity of the deadline values, realtive mix of hard and soft real-time messages, and priority levels of the messages.
    \item Real-time is critical need for many computing and communication applications (form QoS needs).
    \item Real-time scheduling process issues addressed in ref 6.
    \item Notion of deadline and real-time network traffic management in ref 3.
    \item Delay concept in optical network addressed from a complimentary point of view in ref 2.  Proposes design of optimum TDM schedule for broadcast WDM networks to construct transmission schedules of length contrained by the lower bound of tuning ltencies.
    \item TDM-to-WDM data format conversion in ref 1.
    \item Real-time TWDM network issues discussed in ref 5, proposed a distributed adaptive protocol for deadline critical service on an optical network.  Makes use of a single token circulating through control wavelength for communicating status information between each node and controlling access to each of wavelength.
    \item Assumes a centralized controller for the network which schedules transmission in the network, allocates bandwidth, and perfroms admission control when a new message is generated.
    \item Reference 8 has discussion about network topology and TWDM embedded implementation.
    \item Deadline needs of multiple priority class traffic are traded against each other to best satisfy higher priority messages.
    \item Proposed approach extends to multihop networks has well, analytical techniques compute the optimum number of slots required (minimally for hard RT messages and desirably for soft RT messages) for each real-time message.
    \item Implemented using a heuristic scheduling algorithm with three proposed heuristics to select the message to be preempted on a dynamic basis.
    \item System behavior:
    \begin{itemize}
        \item Source node of the network computes an optimum bandwidth according to message size, end-to-end deadline, and other system data.
        \item Source ndoe sends a request to the centralized controller to transmit a message.  Request includes computed results, priority, and routing path.
        \item Controller may reject request or acknowledge and compute a new schedule, will update all nodes with modified schedule.
        \item Controller computes the optimum number of slots required for meeting the deadline and then decides to allocate bandwidth to the new message or reject.
    \end{itemize}
    \item To improve chances of lower priority messages being successfully transmitted, additional slack is added to the optimum number of needed slots, when preemption occurs only the needed slots for higher priority message are removed and there is still some chance the remaining slots for the lower priority message are sufficient.
    \item Heuristics - ALP (Any lower priority) - Removes any lower priority message and does not consider the message size or exact priority level. Simple, but may waste bandwidth in transmission schedule when swapping one or several large bandwidth messages.  LPF (lowest priority first) identifies the message with lowest priority and preempts it.  Similar to ALP with issues but may also cause starvation.  Searching lowest priority message may have overhead. CSS (Closest slot size) preempt messages with lower priority and closest number of required slots.  Reduces the number of messages preempted, also has setbacks.
    \item Analysis fixes a new message generation probability and sweeps the average deadline and computes the successful transmission rate.
    \item
\end{itemize}


\section{Routing and Time-Slot Assignment in Photonic Circuit Switching Networks}
Wing Wa Yu, Albert K. Wong, C.-T. Lee

\subsection{Thoughts}
\begin{itemize}
    \item Minimizing buffering requirement could be useful for future devices to reduce the number of different paths a node is building links for.  Ie. we don't want to store qubits for different paths at the same time and would prefer to establish both sides of a link as close together in time as possible.  May be relevant in how we want to store the first qubit for as little time as possible.
    \item Primary difference between our problem is that packets need to flow in path order in classical networks whereas links in quantum networks can be built in an order along the path.
    \item Another heuristic that is worth exploring is minimizing the timespan of the set of the selected slots.
    \item Best slot method for slot assignment resembles current brute forcing algorithm with extension that schedules are searched with links in every possible order rather than first available slot in the forward direction.
    \item Random method may be useful also for removing the overhead of searching the best slot for all initial slots.
\end{itemize}

\subsection{Notes}
\begin{itemize}
    \item Objective to explore PCS as a method to sub-divide individual wavelengths. Consider ways to route and select time slots for photonic circuits so that buffering requirement in the network can be minimized.  Three heuristic models presented.
    \item Routing and Time slot Assignment problem is similar to Wavelength Routing and Assignment for WDM networks.
    \item Used heuristic aims to minimize total buffering delay.
    \item First routing algorithm is based on the time slot schedules used at the nodes.  Shortest path (Dijkstra) using the reciprocal of the residual bandwidth (available time slots) of each link as the link cost.  Two different slot assignment schemes are then used.  Random method selects one of the incoming slots that is available and then takes the first available slot in the subsequent links.  Best slot method searches all available slots of the first link and sees which one has minimum total buffering delay (uses first available slot on each subsequent link).
    \item Second routing algorithm uses path vector routing.
\end{itemize}


\section{Study on the Problem of Routing, Wavelength and Time-slot Assignment toward Optical Time-slot Switching Technology}
Shan, Dai, Sun, Zhu, Liu

\subsection{Thoughts}
\begin{itemize}
    \item Consideration for slot-size is a good discussion point, guardtime needs to be taken into account for response times of devices operating at the end nodes.
    \item
\end{itemize}

\subsection{Notes}
\begin{itemize}
    \item Existing transmission control protocols and scheduling algorithms for OBS networks lack ways to deal with QoS guarantee and contention avoidance.  Examples include JET (ref 1/2), JIT (ref 4), Horizon scheduling.
    \item One-way resource reservation mechanisms and uncertain burst size are incapable of providing connection-oriented services.
    \item Define $T$ as the average duration of a connection, if the time-slot size is fixed, the $T$ will be a constant time, $\frac{1}{T}=\mu$ will be the rate of service.
    \item Designing of time-slot size and frame length should be taken into account carefully, guardtime is needed to defend optical switch operation, current response time of optical switch is from several nanoseconds to hundreds of nanoseconds
    \item p-distribution approach tries to spread out subsequent slots by occupying the index of the next available slot according to a probability distribiution.  distribution function can be modified based on the priority of the traffic.
\end{itemize}


\section{Scheduling Multirate Sessions in Time Division Multiplexed Wavelength-Routing Networks}
Subramaniam, Harder, Choi

\subsection{Thoughts}
\begin{itemize}
    \item Timing synchronization for All Optical Network testbed in ref 4 with 250 microsecond frames with 128 slots is impressive.  The techniques used here may be used to argue feasibility of using short frame/slot lengths in our work?
    \item My current work focuses on scheduling links to optimize the fidelity, but that doesn't seem to be what we want from such a scheduler.  Here, network throughput is optimized.  Maybe we should also focus on maximizing the network throughput while maintaining fidelity at least what is requested.
    \item This paper considers two scheduling strategies, 1) contiguous assignment of all slots of a session on one wavelength (sounds like 1 comm q  and 1 storage q at all nodes), and 2) noncontiguous assignment of a session's slots on possibly multiple wavelengths (multiple comm/storage qubits).
    \item Work has some nice proofs worth referencing.
\end{itemize}

\subsection{Notes}
\begin{itemize}
    \item Address off-line multirate session scheduling problem, ie the problem of assigning time slots and wavelengths to a given static set of multirate sessions in ring topologies.
    \item Given a set of sessions and their relative rates, objective is to maximize network throughput.  This translates to the problem of minimizing the maximum length of a TDM frame over all wavelengths.
    \item Present a scheduling algorithm with provable worst-case bounds on frame length for multirate session scheduling.  Show off-line single-rate session scheduling problem is equivalent to the off-line wavelength assignment problem and hence obtain bounds on frame length.
    \item Share sessions on the same wavelength using TDM rather than giving a session the full bandwidth of a wavelength.
    \item In a circuit-switched network, slots are assigned during session establishment, and therefore the cycle of switching patterns for each node is known once the sessions are established.  Routing nodes can be programmed to follow this cycle of patterns for each frame.  Allowing guard times between time slots and increasing the duration of slots can be used to mitgate the effects of slow reconfiguration times (ref 3).
    \item Feasibility of timing synchronization for the TDM scheme has been demonstrated in the metropolitan area by the All Optical Network testbed (ref 4).  Each from is 250$\mu s$ long and consists of 128 slots.
    \item Objective function is to maximize network throughput.  Given long-term traffic demands of sessions, it is reasonable to provision the network bandwidth in a fair manner (respect the relative rates of sessions) and maximize bandwidth utilization.
    \item Assignment algorithms and performance analysis for a static traffic model can provide insights into the design and performance of algorithms for the dynamic traffic case.
    \item Study ring topology for "practical suitability in metropolitan area networks".
    \item Prior work in ref 4-7.
    \item In either scheduling case "no two sessions may be assigned the same time slot on the same wavelength if their paths share a link".
\end{itemize}


\section{An Optimal Algorithm for Time-slot Assignment in SS/TDMA Satellite Systems}
Wan, Shan, Shen

\subsection{Thoughts}
\begin{itemize}
    \item One optimization may be taken from the work of Bonuccelli et al.  Schedule long path connections first and then direct links in the gaps.
    \item Satellite systems may be a good reference point for how communication needs to behave.  (ref 1, 5, 10, 11, 12)
    \item Satellite Switched TDMA (SS/TDMA)
    \item Optimal in the sense that the length transmission schedule is minimized.
    \item Methodology for scheduling may be useful, ref 13 for BCW algorithm to decompose traffic matrix into a set of switching matrices. Network flow algorithm in ref 14, push-relabel method.
    \item Birkhoff-Von Neumann theorem for $L(R)$.
\end{itemize}

\subsection{Notes}
\begin{itemize}
    \item Revisit Maximum Traffic Time-slot Scheduling (MTTS) for both circuit- and packet-switched traffic in SS/TDMA satellite systems.
    \item Scheduled circuit-switched traffic first into TDMA frames and then inserted packets into empty frames.
    \item Referred to as Incremental Time-slot Assignment (ITSA) Problem and proved to be NP-hard.
    \item MTTS problem is NOT NP-hard, optimal algorithm based on aggregating the traffic into scheduled TDMA frames based on network flow method.
    \item MTTS seeks for an assignment of maximum packets from packet-switch traffice to circuit-switched traffic such that these packets can be transmitted with circuit-switched traffic together.
    \item Traffic matrix $R$ as an $N \times N$ matrix with non-negative integers.  Each entry $r_{ij}^R$ defines the number of packets to be transmitted from input port i to output port j in a switch.  $|R|=\sum_{i,j} r_{ij}$ is the total number of packets in a traffic matrix $R$. Duration of traffic matrix is the maximum sum of any column or row. Consider traffic matric $P_c$ of circuit-switched traffic and $P_p$ of packet-switched traffic.
    \item Switching matrix $S$ is a traffic matrix that has at most one entry in any row or column.
    \item The length of a schedule $L(R)$ denote the duration fo the traffic matrix where $L(R)$ is the maximum sum along any column or row in the traffix matrix. ($max(max_j(\sum_i R_{ij}), max_i(\sum_j R_{ij}))$).
    \item A transmission schedule in this case is optimal if the length of the schedule is minimized.
    \item Technique of old algorithm
    \begin{itemize}
        \item Construct schedule for static circuit-switched traffic $P_c$ by decomposition of $P_c$ into a set of switching matrices $S_1, S_2,...,S_m$ using BCW algorithm (ref 13.) such that $S_1 + S_2 + ... + S_m = P_c$.
        \item For each switching matrix $S_i$, fill gaps (unused slots) with packets from $P_p$.  ITSA problem was to determine a set of packets $R_p$ selected from $P_p$ and insert these packets into at least one of the switching matrices such that the new switching matrices can still be scheduled in $L(P_c)$ time slots and $|R_p|$ is maximized.
        \item These switching matrices are fixed and causes the problem to be NP-hard.
    \end{itemize}
    \item Optimal Algorithm
    \begin{itemize}
        \item Main difference for MTTS and ITSA is that the switching matrices obtained from decomposing $P_c$ do not have to remain fixed.
        \item Start by turning $P_p$ into a directed bipartite graph $G(U,V,E)$ where nodes $u_i$ corrrespond to rows of $P_p$ and nodes $v_i$ correspond to columns of $P_p$.  Edge set $E$ consists of an edge between $u_i$ and $v_i$ if $d_{ij}^{P_p} \neq 0$. $d_{ij}^{P_p}$ is the entry of $P_p$ in the $i$th row and $j$th column.
        \item Set $D=L(P_c)$.  For each row $i$, compute row vacancy degree $e_i=max\{0, D-\sum_j d_{ij}^{P_c}\}$ and column vacancy degree $f_i = max\{0, \sum_i d_{ij}^{P_c}\}$
        \item Add a source node $T$ and a sink node $K$ to the graph $G(U,V,E)$.  Construct a flow network $G'$ by connecting the source node $T$ to each $u_i$ and connecting each $v_i$ to sink node $K$.  Set the capacity of edge $(T,u_i)$ and $(v_i,K)$ as $e_i$ and $f_i$ respectively.
        \item Apply network flow algorithm (ref 14.) on constructed graph $G'$ to find maximal network flow $f$
        \item Maximal flow $f$ indicates packets that can be promoted to matrix $P_c$. Define $f_{ij}$ as the resulting maximal flow on edge $(u_i, v_j)$.  Particularly, a flow $f_{ij}$ on edge $(u_i, v_j)$ corresponds to promote $f_{ij}$ packets in the row $i$ and column $j$ of matrix $P_c$ to $P_p$.
        \item Construct an $N \times N$ matrix $P_s$, the entry of which on row $i$ and column $j$ is $f_{ij}$, if $f_{ij} exists, otherwise 0$.
        \item Return $P_s$.
        \item $P_s$ is the set of packets in $P_p$ that can be promoted to circuit-switched traffic, the new set of packets is $P_c + P_s$ which is still slotted in $L(P_c)$ time and $P_s$ is maximized.
        \item BCW is used to decompose $P' = P_c + P_s$ into switching matrices that can then be scheduled without conflicts.
    \end{itemize}
    \item Also provide an extension to the MTTS problem where packet-switched data has a deadline after circuit-switched and everything needs to be scheduled.
\end{itemize}


\section{Practical Fast Scheduling and Routingover Slotted CSMA for Wireless Mesh Networks}

\subsection{Thoughts}
\begin{itemize}
    \item Scheduling problem formulation and reduction to PMAX-SAT may be a useful way to see if our problem is NP-Hard.
\end{itemize}

\subsection{Notes}
\begin{itemize}
    \item Focus on wireless mesh networks.
    \item Attempts to come up with a CSMA TDMA schedule to minimize itnerference between transmissions.
    \item Show that the problem is NP-hard and reduce to PMAX-SAT.
\end{itemize}


\section{Scheduling Techniquesfor Hybrid Circuit/Packet Networks}
Liu, Mukerjee, Li, Feltman, Papen, Savage, etc

\subsection{Thoughts}
\begin{itemize}
    \item Hybrid network approach could be a useful way to tackle applications that need lots of entanglement close in time (high-rate) and applications with small amounts of entanglement (low-rate).
    \item Methodology to use an optimal scheduler to gain insight on an effective heuristic may be a good approach to talk about.
    \item Should take into consideration the type of traffic that is flowing through the quantum network to see if there are any properties that could be taken advantage of (like sparsity and skew here for datacenter networks).
\end{itemize}

\subsection{Notes}
\begin{itemize}
    \item Hybrid circuit/packet switched networks place large traffic demands via high-rate circuits and remaining traffice with lower-rate traditional packet-switches.
    \item High utilization requires an efficient scheduling algorithm that can compute proper circuit configurations and balance traffic across the switches.
    \item Formalize hybrid switching problem, explore design space of scheduling algorithms, and provide insight on using such algorithms in practice.  Propose a heuristic-based algorithm Solstice.  Show that it is within 14\% of optimal at scale.
    \item Computing an optimal set of circuit configurations to mazimize circuit-switched utilization has no known polynomial time algorithms, scaling as $O(n!)$ in the number of switch ports (3).
    \item Shed light on optimal/impractical solution that sheds light on how to design an effective heuristic
    \item Algorithm provides higher utilization by taking advantage of the known sparsity and skew of datacenter workloads.
\end{itemize}


\section{A Survey of Network Design Problems and Joint Design Approaches in Wireless Mesh Networks}

\subsection{Thoughts}
\begin{itemize}
    \item Papers for first generations scheduling algorithms on simplified graphs might be useful.
    \item CSMA techniques might be useful, can "ping" a heralding station to detect when the link medium is free.
    \item By "hardwiring" certain repeater protocols along subpaths the requirements for "relatively stable" demand on the subpath may satisfy requirement for TDMA scheduling and achieving maximum throughput.
    \item For the single qubit case it may be possible to model as a matching problem or set covering problem.
    \item Refs 135 and 136 may provide some ideas on converting a centralized algorithm to a distributed one.
\end{itemize}

\subsection{Notes}
\begin{itemize}
    \item References 49 through 58 may be useful for link scheduling, channel assignment, routing.
    \item Sections 4, 5 and 6 deal with link scheduling, channel assignment, routing.
    \item Link scheduling estimates the interference conflicts between the links having transmissiondemands (based on the interference model) and tries to achieve a conflict-free feasible transmission schedule.
    \item First generation of scheduling algorithms (refs 123-127) based on simplified graph models.
    \item When traffic is relatively stable (non-sporadic), TDMA can achieve maximum system capacity.  Distributed implementation is substantially difficult and requires tight time synchronization.  Relatively inflexible to dynamic changes to the topology.  TDMA protocols can be categorized into course-grained and fine-grained protocols.  Coarse-grained protocols emphasis given to link scheduling with various valid assumptions of interference model, traffic demands, and centralized control.  Depend on existing link layer techologies for framing, acknowledgements, while handling medium access and transmission control at upper layers.  Fine-grained TMDA protocols often handle all link layer functions at MAC layer, which makes then increasingly difficult to implement in practice.
    \item Reference 131 provides an LP formulation for node-based and edge-based spatial reuse TDMA scheduling for physical interference model.
    \item Reference 132 provided traffic controlled schedule generation algorithm.
    \item Important to model interference relationship between links based on respective interference model before they can be scheduled.
    \item Problem of link scheduling can be represented as a problem of finding maximum independent set in the conflict graph.  Vertices connect to each other in the conflict graph represent those links of communication graph which interfere with each other and cannot be scheduled simultaneously.
    \item Reference 50 first designs conflict graph for protocol interference model indicating which set of links interfere with each other and cannot be scheduled together.  Conflict graph in physical interference model has vertices which correspond to edges in communication graph.  Directed edge between two vertices whose weight indicates what fraction of the maximum permissable no;ise at the receiver of one link by activity on another link.
    \item Conflcit graph based on interference model adds interference constraints to the LP formulation which optimizes throughput for single source-destination pair.  LP formulation requires calculating all possible transmission schedules and it is shown to be computationally expensive.
    \item Reference 51 proposes a method to simplify the design of conflict graph in the physical interference model.  The node-based conflict graph is designed by keeping the vertex set same as the communication graph and adding di;recting edges uv between vertices u and v whose weight corresponds to the received power at v from the signal transmitted by u.  Only constraint is that node cannot transmit and receive on different links simultaneously.
    \item So this style forms a matching in communication graph.  Propose an efficient polynomial-time scheduling approximation algorithm.
    \item Becomes increasingly difficult to estimate or even bound the optimal scheduler performance.
    \item Reference 134 derives a column generation method using set covering formulation which efficiently solves the scheduling problem.
    \item References 135 and 136 use netwrok flow problem and also provide a distributed scheduling algorithm.
    \item Channel/radio assignment mechanisms try to assign different non-interfering channels to the interfering links to increase overall spatial reuse.  References 52 and 167 discuss import design issues and practical challenges while designing multi-channel protocols for WMNs.  Channel assignment protocols can be broadly classified in static, dynamic, and hybrid schemes.
    \item Channel assignment problem can be modeled as edge-coloring of the network graph and related well-known heuristics or algorithms can be applied for the solution.
    \item
\end{itemize}


\section{GarQ: An Efficient Scheduling Data Structure for Advance Reservations of Grid Resources}

\subsection{Thoughts}
\begin{itemize}
    \item Discusses time-slotted data structures, segment tree.  Provides a search algorithm for an available interval or nearest one.
    \item May be useful for an efficient implementation.
\end{itemize}

\subsection{Notes}
\begin{itemize}
    \item Propose Grid advanced reservation Queue (GarQ) which is a new data structure that improves some weakness of Segment Tree and Calendar Queue.  Demonstrate superiority of the proposed structure by conducting a detailed performance evaluation on real workload traces.
    \item Advanced Reservations allows users to gain simultaneous and concurrent access to adequate resources for the execution of applications.
    \item Tree-based data structures are commonly used for admission control in network bandwidth reservation (refs 3, 21, 23).
    \item Describe modified versions of Linked List and Segment Tree data structures to support add, delete, and search as well as the interval search operation capable of dealing with advanced reservations in computational Grids.
    \item Model composed of a reservation system responsible for handling reservation queries and requests, a grid consisting of a grid information service, resources, and users.
\end{itemize}

\section{Link Scheduling in Sensor Networks: Distributed Edge Coloring Revisited}

\subsection{Thoughts}
\begin{itemize}
    \item When nodes in a quantum network are not attempting entanglement generation according to a TDMA schedule, they may focus on preserving entanglement or other local computations.
    \item Comparing to the edge coloring problem may be a useful approach.
    \item Problem is that all transmissions are assumed to take the same amount of time (1 slot).  Could slots be made large enough to also have this type of affect.  Could assign an edge multiple colors for each of the timeslots that it occupies?
    \item NP-Completeness of minimum number of timeslots ref 21 could be useful
\end{itemize}

\subsection{Notes}
\begin{itemize}
    \item Consider problem of link scheduling in a sensor network employing a TDMA MAC protocol.
    \item Involves two phases, assign a color to each edge in the network such that no two edges incident on the same node are assigned the same color.  Distributed edge coloring algorithm that needs at most delta+1 colors, where delta is the maximum degree of the graph.
    \item Second phase maps each color to a unique timeslot and attempt to identify a direction of transmission along each edge such that the hidden terminal and exposed terminal problems are avoided.
    \item Reference 5 discusses TDMA MAC protocols.  These eliminate collisions, guarantee fairness and provie bounds on per-hop latency.  To conserve energy, a node employing a TDMA MAC protocol can switch off its transceiver when it is neither transmitting nor receiving.
    \item References 6 and 21 discuss link scheduling.
    \item Edge coloring as a link scheduling problem - Valid timeslot assignment for nodes can be obtained by mapping each timeslot to a color.  In wireless sensor networks this causes collision problems (probably not in quantum).
    \item Timeslot assignment to nodes is referred to as broadcast scheduling (refs 2, 20, 23).
    \item The problem of assigning timeslots to edges using a minimum number of timeslots is known to be NP-complete (ref 21)
\end{itemize}

\section{On the Complexity of Scheduling in Wireless Networks}
Sharma, Mazumdar, Shroff

\subsection{Thoughts}
\begin{itemize}
    \item $K=1$ hop problem may be related to single comm q case?
    \item Ref 30, Tassiulas and Ephremides characterized the capacity region of constrained queueing systems, develop a queue length based scheduling scheme that is throughput-optimal (stabilizes the network if the user rates fall within the capacity region of the network)
    \item References 34, 23, 22, 29, 7 show work in cross-layer optimization algorithms that try to consider multiple criteria like optimal routing, link scheduling, and power control.
    \item Changing conditions in wireless networks may be similar to how quantum devices need to be recalibrated and drift.
    \item References 16, 26, 3, 35 discuss fair resource allocation in wireline networks.
    \item References 4, 19, 21, 33, 28, 36, 24 incorporate congestion control frameworks.  Congestion control component controls the rate at which users inject data into the network so as to ensure that the user rates fall within the capacity region of the network.
    \item Shows a general global optimization problem for schedulers in networks, these are known to be NP-Complete and Non-Approximable.
    \item Consider the Maximum Weighted K-Valid Matching Problems (MWKVMPs) that arise as simplifications to the scheduling problem.
    \item Node exclusive interference model for the single communication qubit case.  Ref 11 has polynomial time link scheduling algorithm.
    \item Refs 4, 5, 19 have distributed link scheduling algorithm
    \item
\end{itemize}

\subsection{Notes}
\begin{itemize}
    \item Consider the problem of throughput-optimal scheduling in wireless networks subject to interference constraints.
    \item Interference modeled using a family of $K$-hop interference models defined as one for which no two links within $K$ hops can successfully transmit at the same time.
    \item For $K=1$ the resulting problem is the Maximum Weighted Matching problem which can be solved in polynomial time.
    \item For $K > 1$ show that resulting problems are NP-Hard and cannot be approximated within a factor that grows polynomially with the number of nodes.
    \item Show that for specific kinds of graphs used to model the underlying connectivity graph of wireless networks the resulting problems admit polynomial time approximation schemes.
    \item Simple greedy matching algorithm provides a constant factor approximation to the scheduling problem for all $K$ in this case.
    \item Capacity of wireless links fluctuate due to fading, changes in power allocation or routing changes.
    \item Node exclusive interference model commonly used model for bluetooth and FH-CMDA networks.  Only constraint on the set of edges scheduled to transmit is that it must constitute a matching.
    \item Reference 11 develops a polynomial time link scheduling algorithm under node exclusive interference model.
    \item References 19, 4, 5 developed distributed schemes that guarantee a throughput within a constant factor of the optimal.
\end{itemize}

\section{Pinwheel scheduling with two distinct numbers}
Holte, Rosier, Tulchinsky, Varvel

\subsection{Thoughts}
\begin{itemize}
    \item Could be used for categorizing different types of traffic and using pinwheel problem.
    \item Other known results, every instance of pinwheel scheduling with density at most 3/4 has a solution, every instance with three distinct repeat times and density at most 5/6 has a solution.  When there exists a solution an upper bound on the period is at most the product of the repeat times (exponential), not always possible to find a sub-exponential length schedule.  With compact representation that shows distinct repeat times and number of objects with that time the pinwheel scheduling is NP-Hard.
\end{itemize}

\subsection{Notes}
\begin{itemize}
    \item Interesting questions - Determining whether schedules exist, minimum cyclic schedule length, creating an online scheduler.
    \item Any instance with density = $\sum_{i=1}^n \frac{1}{a_i} > 1$ cannot be scheduled.  Paper concerns element having only two distinct values.
    \item All such instances with only two values and $d \leq 1$ can be scheduled using a strategy based on balancing.  Schedule created is not always of minimum length however.  More complicated method used to create a minimum-length cyclic schedule.  Both polynomial time algorithms but former much easier to compute than latter.
    \item Show how to use either method to produce a fast online scheduler.
    \item Minimum cycle length may be exponential in the length of the input (ref 2).
    \item Family of hard-real-time scheduling problems in refs 4,5,6,8,9.  Periodic maintenance problem EXACTLY every $a_i$ slots.
    \item Decision and scheduling problems for dense instances of up to three distinct numbers can be solved in polynomial time.  Minimum schedule length is the LCM of the three numbers.
    \item Dense instances have interesting properites: Minimum schedule length for instances that can be scheduled is the LCM of distinct numbers.  Slots assigned to item i must occuir exactly $a_i$ slots apart.  These and related properties do not hold for nondense instances.
    \item Multiset representation $\{x, a, y, b\}$ means there are $a$ elements with frequency $a_i=x$ and $b$ elements with frequency $a_i=y$.  In this case we have $a > 0$, $b > 0$, $\frac{x}{a} > 1$ (because $\frac{a}{x} < 1$) and similarly $\frac{y}{b} > 1$.  If dense then $\frac{a}{x} + \frac{b}{y} = 1$
\end{itemize}

\section{TMDA Scheduling with Maximum Throughput and Fair Rate Allocation in Wireless Sensor Networks}

\subsection{Thoughts}
\begin{itemize}
    \item These types of protocols are called Media Access Control (MAC) protocols
    \item These papers always describe the model they assume for the network
\end{itemize}

\subsection{Notes}
\begin{itemize}
    \item Propose network-wide optimized TDMA scheduling scheme for WSNs
    \item Formulate the rate allocation problem based on the Lexicographic Max Min (LMM) criterion which takes fairness, throughput maximization, and slot reuse into consideration.
    \item Develop a polynomial-time algorithm by exploiting iterative linear program to solve LMM optimization
    \item Maximizing sum of data rates (MSDR) of all nodes can easily lead to a baised rate allocation
    \item Exploit slot reuse to improve maximum throughput, introduce slot reuse control parameter $\epsilon$ in the LMM rate allocation problem.  According to the optimal rate allocation vector and relay scheme derived from the LMM optimization, we then propose a TDMA scheduling algorithm, utilizing slot reuse, to achieve a minimum tDMA frame length.
    \item Slot reuse enables the improvement of the network throughput, the limited network bandwidth should also be considered.  Develop a procedure combining the LMM rate allocation and the TDMA scheduling algorithm to iteratively calculate a proper slot reuse control parameter.
    \item
\end{itemize}

\section{Scheduling Algorithms for TDMA Wireless Multihop Networks Thesis - CONTINUE THIS}
Peter Djukic

\subsection{Thoughts}
\begin{itemize}
    \item Frame utilization is a good metric to look at.
    \item Is there an analogous way to view channel quality as done here?
    \item Notion of a conflict graph in network when resources are/not available for generating entanglement with neighbors.  Thesis shows figures depicting different interference modes.  1) two nodes attempting entanglement with a node that only has one remaining resource, 2). So a static conflict graph might indicate nodes that do not have resources to support generating links on all connected edges, or it can dynamically be generated when assigning slots and showing which links can no longer be supported.
    \item Does routing provide a path or a protocol?  Assume someone gives us a protocol, not an ordered set of edges because protocols may be more complex.
    \item Computation of requested link rates may be slightly different because it does not translate directly into the end-to-end rate of the connection.  Depends on how many links need to be generated as part of the protocol.  This could still be useful.
    \item Number of requested slots depends on the number of bits that can be transmitted per slot, in our case this is most likely less than 1.
    \item One of the main differences between our problem and classical networking scheduling is that packets can be stored and forwarded on any of the scheduled link transmissions in the future, whereas the entanglement generated NEEDS to belong to it's associated generated entanglement so that it can be used properly.
    \item Create new schedule with set of rates, do not try to keep the same slots occupied between frames.
    \item Look into the multi-commodity problem.
    \item Look into mesh networks and 802.16.
\end{itemize}

\subsection{Notes}
\begin{itemize}
    \item Link rates and resulting schedules are determined from end-to-end rates.  Connections request their end-to-end rates and a scheduler grants end-to-end rates through a schedule.  Since a schedule does not change between requests it is possible that schedule is temporarily not matched with the required end-to-end connections. (in 802.16 usually on the order of milliseconds.
    \item Frame Utilization $\rho$ - The ratio of slots carrying traffic to the total number of slots in the data sub-frame.
    \item Goal is to find scheduling algorithms that minimize $\rho$  given a set of end-to-end requested rates.  Consistent with notion of finding minimum length TDMA schedules.  Synonymous with maximizing end-to-end throughput.
    \item Model channel quality using the number of bits transmitted in each slot.
    \item Model conflict graph in the network that keeps track of interference of wireless links.  Triplet $G_c (E,C,f_c)$ where $E$ is the set of edges, $C$ the set of TDMA conflicts, and $f_c$ maps $C$ to a set containing pairs of links that have conflicts.
    \item Assume that a routing protocol establishes routes between nodes and are represented with ordered sets of links (end-to-end paths in the topology graph).
    \item Given the requested end-to-end conection rates, find requested link rates by adding up connections traversing each link.  Represent it with a vector of individual link rates.
    \item The number of requested slots for each link in the frame matches the requested link rate.  Number of slots for each link in a frame matches the requested link rate ($\frac{r_j}{b_j}T_f$ where $r_j$ is rate, $b_j$ is bits transmitted per slot, $T_f$ is the frame duration in seconds.)
    \item Duration of a link's transmission is then the number of slots multiplied by the slot duration.
    \item Link capacity constrains the achievable rate across all connections
    \item DAS problem like, calculate requested link durations based on requested rates, find a vector of starting times such that a schedule exists with the desired rates.  Only tells us if there is/not a schedule.
    \item ML-DAS problem tries to maximize the durations with factor $\alpha \leq 1$ such that a schedule exists.  By maximizing $\alpha$ we maximize the link durations.  A feasible schedule might only be possble if we reduce the durations by some amount, by maximizing them we find the minimum length schedule that can satisfy the desired rates (reduced by the factor $\alpha$).
    \item Techniques do not try to maintain the same set of slots between frames, if there is a new set of links we create a new schedule with the set of rates and send it out, so slot positions may change.
    \item Four reasons to solve the maximum concurrent flow problem, 1) Without delay constraints, problem viewed as the problem of finding the end-to-end "capacity" of wireless networks, 2) Corresponds to the "minimum length" or "maximum utilization" TDMA scheduling problem, 3) Optimization produces end-to-end rates that satisfy a limited proportional fairness property, 4) Method of framing the problem fits nicely into the 802.16 scheduling framework.
    \item Joint rate control and scheduling - previous optimization needs the requested end-to-end rates before it can proceed, requires operator input which is not desirable in mesh networks.  Generalize the problem so that the requested rates are not necessary and at the optimum point, the granted end-to-end rates satisfy a property such as proportional fairness, max-min fairness or maximum profit. Generalize the previous problem by using a utility function associated with the end-to-end rate.
    \item Joint routing and scheduling - Looking for best rates over all paths results in the multi-commodity problem.  This thesis uses a set of fixed paths.  In the multi-commodity problem, there are $p$ commodities corresponding to $p$ end-to-end connections, where it is possible that there are more end-to-end connections than nodes in the network ie $p \geq n$.  Suppose we have a connection $\phi$ with source and destination nodes $v_i$ and $v_j$.  At the source node, $x_i^\phi > 0$ is the number of bits sent to the connection's destination in one frame.  $2 \leq i \leq n$ corresponds to uplink connections and $i=1$ for the downlink connections.  At the destination node, $-x_j^\phi$ is the number of bits reaching the node of the connection in one frame.  $2 \leq j \leq n$ for mesh networks. Multi-commodity problem notoriously hard and solution provides multiple paths which makes routing complicated.  
    \item 802.16 requires a solution that provides a set of routes that form a tree.
    \item TDMA Scheduling - Examine the scheduling problem when the number of slots in each frame on the virtual slots axis is fixed and end-to-end delay is not an issue.  Delay constrained TDMA discussed later.
    \item 
\end{itemize}

\section{A Negotiation Based TDMA Scheduling with Consecutive slot assignments for wireless sensor networks}

\subsection{Thoughts}
\begin{itemize}
	\item Using locations in schedules to share synchronization mechanisms is an interesting aspect for removing clock drift. (This multihop parent-children broadcast synchronization approach is similar to root-neighbors synchronization approach in FTSP [32]. FlexiTP paper - The
FlexiTP time synchronization scheme is desirable because children only need to have the same clocks as their parent to ensure that a parent is in the receive mode when a child sends data to it and vice versa. This local synchronization scheme is simple but effective because clock drift is minimized by synchronizing nodes during each data- gathering cycle. Furthermore, it incurs low overheads since the synchronization message is piggybacked to the MFS packet.)
    \item Diagram style is very nice.
    \item Multi-hop slot propagation might be interesting idea for single communication qubit networks
\end{itemize}

\subsection{Notes}
\begin{itemize}
    \item To avoid transmission collisions between schedules can also propagate slot info to three(+)-hop neighbors,
    \item In this paper, we propose a TDMA scheduling with well-designed slots negotiation and consecutive slots assignment mechanisms. Furthermore, our approach adopts a local time synchronization mechanism to remove the impact of clock drift. In our approach, each node’s schedule consists of slots which are used to data collection and time synchronization, respectively. Slots are assigned to nodes in the initial phase. Once all nodes establish their schedules, they receive and transmit data according to their schedules in each data collection cycle.
    \item 
\end{itemize}

\section{Optimal Timeslot Size for Synchronous Optical Burst Switching}

\subsection{Thoughts}
\begin{itemize}
    \item Explanation of guard times nice.  Important to remember that the slot size has an impact on how much guard time ends up filling the frame.  Trade off with granularity of link usage (bursts) and spending more of frame in guard time.  What if we have mini slots in slots so guard times only occur every $N$ mini slots?
    \item Because link generation is low probability, can we model entanglement generation as "bursty"?
    \item Idea to think of timeslot utilization using probability of generating adjacent links in the network.  There is a point where increasing the slot duration increases the number of links expected to be built in a slot but may lose rate due to switching links less frequently. This suggests that networks with different probabilities of generating elementary links have different optimal slot sizes.
    \item Calculation of burst blocking probability uses equation in reference 2, might be useful for above
    \item 
\end{itemize}

\subsection{Notes}
\begin{itemize}
    \item Guard time used between consecutive slots to accompany device calibration delays
    \item Investigate effect of timeslot size in order to provide an analytical framework for estimating the optimized solution, based on a given network configuration and traffic characteristics/
    \item  Analyzes time slot utilization based on the amount of bursty traffic expected provided a probability distribution.
    \item Weighted Burst Loss Approximation - Compute offered load in the network, approximate offered load on each link, and approximate burst blocking probability at each link.
    \item 
\end{itemize}

\section{Accurate clock synchronization for IEEE 802.11-based multi-hop wireless networks}
\subsection{Thoughts}
\begin{itemize}
    \item Interesting, might be useful for some protocol specific things.
\end{itemize}


\subsection{Notes}
\begin{itemize}
    \item IEEE 802.11 uses a 64-bit TSF counter with micro-second resolution for synchronizing nodes in wireless network operating in ad-hoc mode.  Nodes broadcast their TSF value during beacon periods and update local TSF counter to the latest one seen.
    \item NTP most commonly used IEEE standard
    \item 
\end{itemize}

\section{Optimization based rate allocation and scheduling in TDMA based wireless mesh networks}
\subsection{Thoughts}
\begin{itemize}
    \item Network Utility Maximization (ref 5), proposed for internet congestion control (ref 14, 15), important tool for modeling and designing resource allocation algorithms
    \item TDMA MAC protocol ref 6
    \item CSMA style protocol not chosen due to difficulty in QoS support
    \item Ref 19 contention models
    \item References 13 and 19 using contention graphs $G_C(V_C, E_C)$ where $V_C$ contains all transmissions in the network and $E_C$ indicates that two vertices contend with each other.
    \item References 12 and 20 for clique/independent set stuff.
    \item Reference 7 Least Overlapped First scheduling algorithm
    \item Figure style is nice for schedules
    \item Look at the algorithm more
\end{itemize}

\subsection{Notes}
\begin{itemize}
    \item Propose a framework that performs both rate allocation and scheduling for unicast and multicast traffic in TDMA-based wireless mesh networks.
    \item Rate allocation algorithm based on Network Utility Maximization.  Graph coloring-based scheduling algorithm achieves the allocated rates.  
    \item Scheduling algorithm based on graph coloring of the contention graph and assumes the spatial TDMA MAC protocol
    \item Provides efficient and fair rate allocation for both unicast and multicast traffic, framework effectively schedules the allocated rates which results in guaranteed throughput and bounded delay for session recipients, efficient and suitable for centralized impelementation
    \item Rate allocation takes variable link capacities into consideration
    \item By using perfect contention graphs to model rate allocation constraints, guaranteee that the allocated rates can be supported by an efficient scheduling algorithm proposed
    \item Define notion of $transmission$, which is a generalization of a link.  A transmission consists of a sender, list of recipients, session carried by transmission, transmission rate used.
    \item Transmission Contention Graph - Ref 19 - Protocol Model and Physical Model.  Example of Protocol Model in ref 8. Under this model every link-layer transmission rate corresponds to a fixed transmission range and interference range.  A transmission $i$ conflicts with transmission $j$ if 1) Both transmissions share the same sender, or 2) any recipient of transmission $j$ is within the interference range of the sneder of transmissoin $i$ or vice versa.
    \item Define a transmission contention (conflict) graph based on the contention relationship among different transmissions.  Contention graphs have been used in several papers to model network resource constraints.
    \item Maximal cliques and maximal independent sets of contention graphs are interesting.  A set of nodes is a clique if its induced subgraph is complete.  A maximal clique is defined as a clique that is not contained in any other clique.  Denote the set of maximal cliques of a contention graph as $C$.  An Independent Set is a set of nodes that are not connected by any edges, and a maximal independent set is an independent set not contained by any other independent set.
    \item For a contention graph $G_C$, $I$ is the family of all independent sets of the graph.  Schedule $D$ can be defined as an infinite sequence of independent sets $I_1, I_2,...,I_k,...$ where $I_k \in I$.
    \item Compute a frequency of transmission and claim that it can be treated as the normalized bandwidth allocated to a specific transmission.  A vector of frequencies is clique feasible for the contention graph if $\sum_{i \in C_j} f_i \leq 1, \all C_j \in C$
    \item Has been shown that if the contention graph is a perfect graph, clique feasibility is equivalent to scheduling feasibility of the frequency vector.
    \item A perfect graph is a graph in which every induced subgraph can be colored with $p$ colors where $p$ is the size of the largest clique in the subgraph,  A chordal graph is a graph that does not contain an induced $k-$cycle for $k \geq 4$ is an instance of a perfect graph. 
\end{itemize}

\section{On the Scalability of IEEE 802.11 Ad Hoc Networks}
\subsection{Thoughts}
\begin{itemize}
    \item Might provide some insights into how clock synchronization done in 802.11 and what could be limitations for assuming so in the quantum networks.
    \item Reference 7 does an analysis of the effect of network size and traffic patterns on the capacity of ad hoc wireless networks, might be useful to apply to this MAC protocol.
    \item Lamport also proposes a synchronization mechanism in reference 6
\end{itemize}

\section{Link Scheduling Algorithms for Wireless Mesh Networks}
\subsection{Thoughts}
\begin{itemize}
    \item All papers begin by specifying a model of the network, should do this for work.
    \item Definitions for protocol interference and physical interference model definitions in reference 19 could be a good guide to defining a new interference model for quantum networks that motivates link scheduling.
\end{itemize}

\subsection{Notes}
\begin{itemize}
    \item Reference 18 shows optimal link-scheduling is NP-hard, also shows a graph coloring algorithm
    \item Protocol interference and physical interference models.
\end{itemize}

\section{A dynamic adjustable contention period mechanism and adaptive backoff process to improve the performance for multichannel mesh deterministic access in wireless mesh LAN}

\subsection{Thoughts}
\begin{itemize}
    \item A contention-based mechanism can be formulated by simply sending the control messages to the midpoint without actually performing the photon-emission.  Can split time into a contention period where everyone is trying to attempt entanglement generation using a midpoint but only begins photon emission after the contention period is over, maybe midpoint can be involved in deciding which end node(s) get to go-ahead with entanglement generation?  Avoids unecessary entanglement attempts but introduces additional noise on the qubits due to duration of contention period.
    \item This work uses the IEEE802.11 ad-hoc timing synchronization mechanism in reference 30.
    \item Distributed coordination function (DCF)?  Used at beacon intervals?
\end{itemize}

\subsection{Notes}
\begin{itemize}
    \item Contention mechanism shown in Figure 2 may be a useful idea for a CSMA-like MAC protocol.
    \item Mesh point sends beacon in accordance with the principles of the distributed coordination function (DCF) at the beginning of each beacon interval.  Other MPs cease sending beacons and take the timestamp values in the beacons if any MP sends the bacon successfully.
\end{itemize}

\section{An On-demand and Dynamic Slot Assignment Protocol for Ad Hoc Networks}

\subsection{Thoughts}
\begin{itemize}
    \item Argue that applying GPS or time analysis schemes in reference 1 can be used for slot synchronization.
    \item Figures on different reservation collisions may be useful for thesis.
    \item Theoretical performance analysis based on slot collisions may be useful.
\end{itemize}

\subsection{Notes}
\begin{itemize}
    \item Dynamic slot assignment is difficult to realize as ad hoc networks are distributed and their topologies are variable.
    \item Several slot assignment protocols for ad hoc networks have been proposed which can be classified into three classes.  Absolute toplogoy transparent (like TSMA in ref 2), insensitive to toplogy variation (like FPRP, Five Phase Reservation Protocol, in 3, ref 4 also has something), or assigning slot to node based on its two hop neighbor information (like USAP, Unifying Slot Assignment Protocol, in references 6 and 7).
    \item Most current slo tassignment schemes for ad hoc networks are inefficient in adapting to variations of applications and network toplogy like TSMA and USAP,FPRP has partitioned TDMA frames that contain a reservation subframe an dinformation subframe.  These subframes interleaved periodicallywhich allows for slot competition amongst neighbors and to inform neighbors of competition results.  Winners are allowed to use corresponding channels.
    \item Proposed protocol (ODSA) decreases slot reservation collisions by using randomness in slot selection.  On top of this, it should be able to adapt to variations in time delay and ensure time delay which is important for ad hoc networks with real-time applications.
\end{itemize}

\section{An Evolutionary-Dynamic TDMA Slot Assignment Protocol for Ad Hoc Networks}

\subsection{Thoughts}
\begin{itemize}
    \item Discussion of frame format (length, organization) and reservation negotiation should go into thesis.
    \item Discusses a distributed approach where frames have slots that are reserved for reservation negotiation.  This isn't necessarily needed in the link scheduling protocol in the distributed case since we assume the scheduled resource is quantum and not classical.
    \item Also discusses how nodes join the network and what information they obtain.  Nodes joining a quantum network would contact a centralized controller to obtain frame/slot information.
\end{itemize}

\subsection{Notes}

\section{Network Flows Theory, Algorithms, and Applications}

\subsection{Notes}
\begin{itemize}
    \item Section 3.5 - Flow decomposition algorithms - Maybe not too useful
    \item Section 5.1 - Label Correcting Algorithms
    \item Section 6 - Maximum Flows
    \item Section 8 - Maximum Flows,  Additional Topics
    \item Section 9 - Minimum Cost Flows
    \item Section 12 - Assignments and Matchings
    \item Section 14 - Convex Cost Flows
    \item Section 15 - Generalized Flows
    \item Section 17 - Multicommodity Flows
    \item Section 19.6 - Dynamic Flows
\end{itemize}

\subsection{Thoughts}
\begin{itemize}
    \item Flow Decomposition
    \begin{itemize}
        \item Formulating network flow problems can be modeled using arcs or flows on paths and cycles.
	\item Arc flow formulation means we have a vector $x=\{x_{ij}\}$ that satisfies:
		$$\sum_{\{j:(i,j) \in A\}} x_{ij} - \sum_{\{j:(j,i) \in A\}} x_{ji} = -e(i) \text{   for all } i \in N$$
		$$0 \leq x_{ij} \leq u_{ij} \text{   for all } (i,j) \in A$$
        \item This is to say that the imbalance of a node is equal to the difference in it's inflow and it's outflow.  An imbalance greater than 0 corresponds to an excess node and an imbalance less than zero corresponds to a deficit node.  Zero is a balanced node.
	\item Flow Decomposition Problem - Every path and cycle flow has a unique representation as nonnegative arc flows.  Conversely, every nonnegative arc flow x can be represented as a path and cycle flow (though not necessarily unique) with the following properties:
	\begin{itemize}
	    \item Every directed path with positive flow connects a deficit node to an excess node.
            \item At most n+m paths and cycles have nonzero flow; out of these, at most m cycles have nonzero flow.
	\end{itemize}
    \end{itemize}
    \item Label Correcting Algorithms
    \begin{itemize}
        \item 
    \end{itemize}
\end{itemize}


\section{Real-Time Systems Lecture Notes - Giorgio Buttazzo}

\subsection{Thoughts}
\begin{itemize}
    \item Task definition useful
    \item Can model idle behavior in protocols at nodes as "busy-wait" times where concurrent actions can be placed into the schedule (lecture 2, page 12 of 15).
    \item Mention of concurrency and conflicts with shared resources might be useful when considering interleaving protocols and having enough resources to support executing them concurrently?
    \item Our problem seems Nonpreemptive, Static, Offline, Use of Optimal vs Heuristic is explored
    \item Possible heuristics: Shortest Job First (Maybe shortest protocol first?), Priority Scheduling,
    \item Scheduling algorithms also accompanied with a feasibility test to make sure schedule is possible.
    \item Deadlines of individual bell pairs may be related to whether they need to be available at the same time or not
\end{itemize}

\subsection{Notes}
\begin{itemize}
    \item Task definition - Has an activation time $a_i$, a start time $s_i$, a computation time $C_i$ and a finish time $f_i$.  The response time of a task $R_i=f_i-a_i$.
    \item A Real-Time Task is a task characterized by a timing constraint on its response time, called a deadline $d_i$. Relative deadline $D_i$ from activation time $a_i$. A task is feasible if it is guaranteed to complete within its deadline $d_i$ or equivalently the response time is less than the relative deadline $R_i \leq D_i$.
    \item Slack corresponds to the amount of time between $f_i$ and $d_i$ if $d_i - f_i > 0$.  If finish time after deadline then this is called lateness $f_i - d_i$.
    \item There are time driven tasks (periodic) and event-driven tasks (aperiodic).
    \item Periodic tasks have activation times $a_{i,k} = \Phi_i + (k-1)T_i$ where $k$ enumerates the instance of periodic task $i$ with periodicity $T_i$.  The deadlines $d_{i,k}=a_{i,k} + D_i$ and often $T_i=D_i$.
    \item Hard tasks must meet deadlines, Firm tasks become invalid if they miss deadline, Soft tasks has flexible deadlines.
    \item Precedence constraints impose an ordering in the task execution.
    \item If a task must wait for I/O or data concurrency allows another task to run during that interval.
    \item Concurrency becomes superior when managing periodic tasks at different rates (using rate monotonic). But can introduce conflicts when using shared resources.
    \item Some simplifying assumptions: Single Processor, Homogenous Task Sets, Fully Preemptive Tasks, Simultaneous Activations, No precedence constraints, No resource constraints.
    \item Descriptions of Preemptive/Nonpreemptive, Static/Dynamic, Online/Offline, Optimal/Heuristic
    \item Optimality Criteria: Minimizing maximum lateness, minimizing number of missed deadlines, Maximizing cumulative value of tasks assigned with values
    \item Discussion of handling precedence requests (lecture 3)
    \item Examples of feasible/infeasible schedules using different algorithms
    \item
\end{itemize}

\section{Deterministic Networking Problem Statement RFC 8557}

\subsection{Thoughts}
\begin{itemize}
    \item Super helpful
\end{itemize}

\section{Deterministic Networking (DetNet) Controller Plane Framework draft-malis-detnet-controller-plane-framework-02}

\subsection{Thoughts}
\begin{itemize}
\end{itemize}

\subsection{Notes}
\begin{itemize}
\end{itemize}

\section{}

\subsection{Thoughts}
\begin{itemize}
\end{itemize}

\subsection{Notes}
\begin{itemize}
\end{itemize}

\section{}

\subsection{Thoughts}
\begin{itemize}
\end{itemize}

\subsection{Notes}
\begin{itemize}
\end{itemize}
\end{document}
